首先,它们不叫\href{https://en.wikipedia.org/wiki/Functor}{functors}。这是一个定义明确的术语,来自数学的一个分支,叫做\href{https://en.wikipedia.org/wiki/Category_theory}{category理论}。

函数对象是行为类似于函数的对象,通过实现函数调用操作符来实现。由于函数对象是对象,若依具有属性,也可以具有状态。

\begin{lstlisting}[style=styleCXX]
struct Square{
	void operator()(int& i){i= i*i;}
};

std::vector<int> myVec{1, 2, 3, 4, 5, 6, 7, 8, 9, 10};

std::for_each(myVec.begin(), myVec.end(), Square());

for (auto v: myVec) std::cout << v << " "; // 1 4 9 16 25 36 49 64 81 100
\end{lstlisting}

\begin{tcolorbox}[breakable,enhanced jigsaw,colback=blue!5!white,colframe=blue!75!black,title={实例化函数对象}]
算法中使用函数对象(Square)的名称而不是函数对象(Square())本身的实例是一个常见的错误:std::for\_each(myVec.begin(), myVec.end(), Square)。必须使用实例:std::for\_each(myVec.begin(), myVec.end(), Square())
\end{tcolorbox}