
在为C++20提供理论之后,我现在将理论应用于实践,并为您提供一些研究案例。

想要多次同步线程时,可以使用条件变量、std::atomic\_flag、std::atomic<bool>或信号量。在“线程的快速同步”部分,我想回答哪种方式是最快的?关于协程的部分,给出了三个基于co\_return、co\_yield和co\_await的协程。我使用这些协程作为进一步实验的起点,以加深对协程控制流的理解。在“Future的变体”章节中,我在使用co\_return时,实现了惰性future和基于future的future。线程的章节中,修改和泛化改进了使用co\_return的生成器。最后,“不同的工作流”章节中从co\_await开始,讨论了一些作业工作流。


