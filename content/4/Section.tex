Congratulations! When you read these lines, you have mastered the challenging and thrilling C++20 standard. C++20 is a C++ standard that likely has the same influence for C++, such as the other two significant C++ standards: C++98 and C++11. Due to C++11, the following names for the C++ standards are used by the C++ community.

\begin{itemize}
\item 
Legacy C++: C++98, and C++03

\item 
Modern C++ : C++11, C++14, and C++17

\item 
<Placeholder>: C++20
\end{itemize}
	
I’m not sure what name will be used for C++20 in the future. I’m only sure that C++20 starts a new C++ area. Let me remind you why, in particular, the Big Four change the way we program in C++.

\begin{itemize}
\item 
Concepts: Concepts revolutionize the way we think about and write generic code. Thanks to them, we can reason about our program for the first time in semantic categories such as Number or Ordering.

\item 
Modules: Modules are the starting point of software components. Modules help overcome the deficiencies of legacy headers and macros.

\item 
Ranges: The ranges library extends the Standard Template Library with functional ideas.
Algorithms can operate directly on the containers, can be evaluated lazily, and can be composed.

\item 
Coroutines: Thanks to coroutines, asynchronous programming becomes a first-class citizen in C++. Coroutines transform blocking function calls in waiting and are highly valuable in event-driven systems such as simulations, servers, or user interfaces.
\end{itemize}

C++20 is just the starting point. There is work to be done in C++23 to fully integrate and use the potential of the Big Four in C++. Let me give you a few ideas about the near C++ future.

\begin{itemize}
\item 
The Standard Template Library was designed by \href{https://en.wikipedia.org/wiki/Alexander_Stepanov}{Alexander Stephanov} with concepts in mind.
Still, the integration of concepts is missing in C++20.

\item 
We can expect a modularized Standard Template Library and hope for a packaging system in C++.

\item 
Many algorithms known from functional programming are still missing in the ranges library.
A future C++ standard should improve the interplay of the range algorithms and the standard containers.

\item 
We don’t have coroutines. We only have a framework for building powerful coroutines. A coroutines library will be, with high probability, in C++23.
\end{itemize}

In the chapter about C++23 and Beyond, I give more details on the near future of C++.
To make it short: C++ has a bright, shiny future.






