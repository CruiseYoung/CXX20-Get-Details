

阶段1的目标是针对特定领域的调查和孵化。研究小组的成员通过面对面的会议,也会在会议间隙,通过电话或视频会议进行交流。中央小组可审查各研究小组的工作,以确保一致性。

这些是领域特定的研究小组:

\begin{itemize}
\item 
SG1,并发性(Concurrency):并发性和并行性主题,包括内存模型

\item 
SG2,模块(Module):模块相关的主题

\item 
SG3,文件系统

\item 
SG4,网络(Networking):网络库开发

\item 
SG5,事务性内存(Transactional Memory):用于添加的事务性内存结构

\item 
SG6,数字相关(Numerics):数字主题,如定点数、浮点数和分数

\item 
SG7,编译时编程(Compile time programming):泛型(编译时)编程

\item 
SG8,概念(Concepts)

\item 
SG9,范围(Ranges)

\item 
SG10,特性测试(Feature Test):用于测试C++特性是否支持,以及可移植性检查

\item 
SG11,数据库(Database):与数据库相关的库接口

\item 
SG12,未定义行为和漏洞(UB\&Vulnerabilities):针对标准中的漏洞和未定义/未指定行为的改进

\item 
SG13,HMI \& I/O(人/机交互接口):支持输出和输入设备

\item 
SG14,游戏开发和低延迟(Game Development \& Low Latency):游戏开发人员和(其他)低延迟编程

\item 
SG15,工具(Tooling):开发人员工具,包括模块和包

\item 
SG16,Unicode: C++中的Unicode文本处理

\item 
SG17,EWG孵化器(Incubator):关于核心语言演化的早期讨论

\item 
SG18,LEWG孵化器(Incubator):关于库语言演化的早期讨论

\item 
SG19,机器学习:人工智能(AI)的特定主题,也包括线性代数

\item 
SG20,教育(Education):C++教材指南

\item 
SG21,契约(Contract):契约式设计的语言支持

\item 
SG22,C与C++的兼容(C/C++ Liaison):讨论C和C++的协调兼容问题
\end{itemize}

本节简要介绍了C++的标准化,特别是C++委员会的标准化。读者们可以在\url{https://isocpp.org/std}上找到更多关于标准化的信息。












