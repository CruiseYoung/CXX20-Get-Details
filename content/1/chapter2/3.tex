

Stage 1 aims for domain-specific investigation and incubation. The study groups’ members meet in face-to-face meetings, between the meeting by telephone or video conferences. Central groups may review the work of the study groups to ensure consistency.

These are the domain-specific Study Groups:

\begin{itemize}
\item 
SG1, Concurrency: Concurrency and parallelism topics, including the memory model

\item 
SG2, Modules: Modules-related topics

\item 
SG3, File System

\item 
SG4, Networking: Networking library development

\item 
SG5, Transactional Memory: Transactional memory constructs for future addition

\item 
SG6, Numerics: Numerics topics such as fixed-point numbers, floating-point numbers, and fractions

\item 
SG7, Compile time programming: compile time programming in general

\item 
SG8, Concepts

\item 
SG9, Ranges

\item 
SG10, Feature Test: Portable checks to test whether a particular C++ supports a specific feature

\item 
SG11, Databases: Database-related library interfaces

\item 
SG12, UB \& Vulnerabilities: Improvements against vulnerabilities and undefined/unspecified
behavior in the standard

\item 
SG13, HMI \& I/O (Human/Machine Interface): Support for output and input devices

\item 
SG14, Game Development \& Low Latency: Game developers and (other) low-latency programming requirements

\item 
SG15, Tooling: Developer tools, including modules and packages

\item 
SG16, Unicode: Unicode text processing in C++

\item 
SG17, EWG Incubator: Early discussion about the core language evolution

\item 
SG18, LEWG Incubator: Early discussions about the library language evolution

\item 
SG19, Machine Learning: Artificial intelligence (AI) specific topics but also linear algebra

\item 
SG20, Education: Guidance for modern course materials for C++ education

\item 
SG21, Contracts: Language support for Design by Contract

\item 
SG22, C/C++ Liaison: Discussion of C and C++ coordination
\end{itemize}

This section provided you a concise overview of the standardization in C++ and, in particular, the C++ committee. You can find more details about the standardization on \url{https://isocpp.org/std}.












